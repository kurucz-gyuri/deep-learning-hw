\documentclass[12pt,conference]{IEEEtran}

\usepackage[colorlinks,urlcolor=blue,linkcolor=blue,citecolor=blue]{hyperref}
\usepackage{color,array}
\usepackage{graphicx}

\setcounter{page}{1}

\begin{document}

\title{%
	Final Report of Homework \\
	\large for the Deep Learning in Practice course \\
	[0.3em] Budapest University of Technology and Economics}

\author{György Józsa, György Kurucz, and Márió Zeller}

\markboth{Journal of IEEE Transactions on Artificial Intelligence, Vol. 00, No. 0, Month 2020}
{First A. Author \MakeLowercase{\textit{et al.}}: Bare Demo of IEEEtai.cls for IEEE Journals of IEEE Transactions on Artificial Intelligence}

\maketitle

\begin{abstract}
The abstract should not exceed 250 words. This example is 250 words. See instructions give you guidelines for preparing papers for IEEE Transactions and Journals. Use this document as a template if you are using Microsoft Word 6.0 or later. Otherwise, use this document as an instruction set. The electronic file of your paper will be formatted further at IEEE. Paper titles should be written in uppercase and lowercase letters, not all uppercase. Avoid writing long formulas with subscripts in the title; short formulas that identify the elements are fine (e.g., ``Nd--Fe--B''). Do not write ``(Invited)'' in the title. Full names of authors are preferred in the author field, but are not required. Put a space between authors' initials. The abstract must be a concise yet comprehensive reflection of what is in your article. In particular, the abstract must be self-contained, without abbreviations, footnotes, or references. It should be a microcosm of the full article. The abstract must be between 150--250 words. Be sure that you adhere to these limits; otherwise, you will need to edit your abstract accordingly. The abstract must be written as one paragraph, and should not contain displayed mathematical equations or tabular material. The abstract should include three or four different keywords or phrases, as this will help readers to find it. It is important to avoid over-repetition of such phrases as this can result in a page being rejected by search engines. Ensure that your abstract reads well and is grammatically correct.
\end{abstract}

\section{Introduction}


\section{Further section}


\subsection{With subsection}

Define abbreviations and acronyms the first time they are used in the text, even after they have already been defined in the abstract. Abbreviations such as IEEE, SI, ac, and dc do not have to be defined. Abbreviations that incorporate periods should not have spaces: write ``C.N.R.S.,'' not ``C. N. R. S.'' Do not use abbreviations in the title unless they are unavoidable (for example, ``IEEE'' in the title of this article).

\section*{References}

\subsection*{Basic format for books:}\vspace*{-12pt}
\def\refname{}
\begin{thebibliography}{34}
\item[] J. K. Author, ``Title of chapter in the book,'' in {\em Title of His Published Book}, xth ed. City of Publisher, (only U.S. State), Country: Abbrev. of Publisher, year, ch. x, sec. x, pp. xxx--xxx.
\end{thebibliography}

\subsection*{Examples:}
\def\refname{}
\begin{thebibliography}{34}\vspace*{-12pt}

\bibitem{}G. O. Young, ``Synthetic structure of industrial plastics,'' in {\em Plastics},\break 2nd ed., vol. 3, J. Peters, Ed. New York, NY, USA: McGraw-Hill, 1964,\break pp. 15--64.

\bibitem{}W.-K. Chen, {\it Linear Networks and Systems}. Belmont, CA, USA: Wadsworth, 1993, pp. 123--135.

\end{thebibliography}

\subsection*{Basic format for periodicals:}\vspace*{-12pt}

\begin{thebibliography}{34}
\item[]
J. K. Author, ``Name of paper,'' {\it Abbrev. Title of Periodical}, vol. {\it x},\break   no. {\it x}, pp. xxx--xxx, Abbrev. Month, year, DOI. \href{https://dx.doi.org/10.1109.XXX.123456}{10.1109.XXX.123456}.
\end{thebibliography}


\subsection*{Examples:}\vspace*{-12pt}

\begin{thebibliography}{34}
\setcounter{enumiv}{2}

\bibitem{}J. U. Duncombe, ``Infrared navigation Part I: An assessment of feasibility,'' {\em IEEE Trans. Electron Devices}, vol. ED-11, no. 1, pp. 34--39,\break Jan. 1959, 10.1109/TED.2016.2628402.

\bibitem{}E. P. Wigner, ``Theory of traveling-wave optical laser,''
{\em Phys. Rev.},  vol.\break 134, pp. A635--A646, Dec. 1965. DOI. \href{https://dx.doi.org/10.1109.XXX.123456}{10.1109.XXX.123456}.

\bibitem{}E. H. Miller, ``A note on reflector arrays,'' {\em IEEE Trans. Antennas Propagat.}, to be published.
\end{thebibliography}


\subsection*{Basic format for reports:}\vspace*{-12pt}
\begin{thebibliography}{34}
\item[]
J. K. Author, ``Title of report,'' Abbrev. Name of Co., City of Co., Abbrev. State, Country, Rep. xxx, year.
\end{thebibliography}

\subsection*{Examples:}\vspace*{-12pt}
\begin{thebibliography}{34}
\setcounter{enumiv}{5}

\bibitem{} E. E. Reber, R. L. Michell, and C. J. Carter, ``Oxygen absorption in the earth’s atmosphere,'' Aerospace Corp., Los Angeles, CA, USA, Tech. Rep. TR-0200 (4230-46)-3, Nov. 1988.

\bibitem{} J. H. Davis and J. R. Cogdell, ``Calibration program for the 16-foot antenna,'' Elect. Eng. Res. Lab., Univ. Texas, Austin, TX, USA, Tech. Memo. NGL-006-69-3, Nov. 15, 1987.
\end{thebibliography}

\subsection*{Basic format for handbooks:}\vspace*{-12pt}
\begin{thebibliography}{34}
\item[]
{\em Name of Manual/Handbook}, x ed., Abbrev. Name of Co., City of Co., Abbrev. State, Country, year, pp. xxx--xxx.
\end{thebibliography}

\subsection*{Examples:}\vspace*{-12pt}

\begin{thebibliography}{34}
\setcounter{enumiv}{7}

\bibitem{} {\em Transmission Systems for Communications}, 3rd ed., Western Electric Co., Winston-Salem, NC, USA, 1985, pp. 44--60.

\bibitem{} {\em Motorola Semiconductor Data Manual}, Motorola Semiconductor Products Inc., Phoenix, AZ, USA, 1989.
\end{thebibliography}

\subsection*{Basic format for books (when available online):}\vspace*{-12pt}
\begin{thebibliography}{34}
\item[]
J. K. Author, ``Title of chapter in the book,'' in {\em Title of Published Book}, xth ed. City of Publisher, State, Country: Abbrev. of Publisher, year, ch. x, sec. x, pp. xxx--xxx. [Online]. Available: http://www.web.com 
\end{thebibliography}


\subsection*{Examples:}\vspace*{-12pt}

\begin{thebibliography}{34}
\setcounter{enumiv}{9}

\bibitem{}G. O. Young, ``Synthetic structure of industrial plastics,'' in Plastics, vol. 3, Polymers of Hexadromicon, J. Peters, Ed., 2nd ed. New York, NY, USA: McGraw-Hill, 1964, pp. 15--64. [Online]. Available: http://www.bookref.com. 

\bibitem{} {\em The Founders Constitution}, Philip B. Kurland and Ralph Lerner, eds., Chicago, IL, USA: Univ. Chicago Press, 1987. [Online]. Available: http://press-pubs.uchicago.edu/founders/

\bibitem{} The Terahertz Wave eBook. ZOmega Terahertz Corp., 2014. [Online]. Available: http://dl.z-thz.com/eBook/zomega\_ebook\_pdf\_1206\_sr.pdf. Accessed on: May 19, 2014. 

\bibitem{} Philip B. Kurland and Ralph Lerner, eds., {\em The Founders Constitution}. Chicago, IL, USA: Univ. of Chicago Press, 1987, Accessed on: Feb. 28, 2010, [Online] Available: http://press-pubs.uchicago.edu/founders/ 
\end{thebibliography}

\subsection*{Basic format for journals (when available online):}\vspace*{-12pt}
\begin{thebibliography}{34}
\item[] J. K. Author, ``Name of paper,'' {\em Abbrev. Title of Periodical}, vol. x, no. x, pp. xxx--xxx, Abbrev. Month, year. Accessed on: Month, Day, year, DOI: 10.1109.XXX.123456, [Online].
\end{thebibliography}


\end{document}
